%{{{ [vim] 
% vim:foldmarker=%{{{,%}}}
% vim:foldmethod=marker
% vim:foldcolumn=3
%}}}
%% preface.tex
%% author: Thanos Tsouanas <thanos@tsouanas.org>

\chapterblah \prefaceterm.

\sectionblah For the reader.

Each theorem proof is marked with ``\thinspace\qedsymbol\thinspace'',
known as a ``Halmos\Halmos[tombstone]~(tombstone)'';%
\footnote{The ideia is that we have killed our goal with our proof,
and so we display its tombstone.}
the solved examples end with ``\thinspace\qexsymbol\thinspace''.
Often, to motivate you to try and prove the theorems before yielding and studying their proofs,
I keep only a rough sketch in the normal text, whose end I mark with
``\thinspace\qessymbol\thinspace''.
In these cases, the full proofs appear in the appendix.
I use ``\thinspace\activitysymbol\thinspace'' to indicate activities in the text
which you should undertake.

For most of the exercises and problems I have multiple hints to help you
reach a solution.  Try attacking them without any hints.
If you cannot make it, look for a first hint in the appendix ``Hints \#1'',
and continue trying.
If it still seems difficult, look for a second hint in ``Hints \#2'', etc., etc.
Finally, when there are no more hints to help you,
there are complete solutions in the appendix.

\emph{Beware!  Never read mathematics passively!}
Some of the proofs have omissions and/or errors
(in these cases ``\thinspace\qedsymbol\thinspace''
turns into ``\thinspace\mistakesymbol\thinspace'').
Problems and exercises ask you to spot those errors,
and---in case the theorems are really provable---you
can find their full and correct proofs also in the appendix.

In some points you'll see ``spoiler alerts''.
This happens when I have just asked you something, or I need
to make a decision, etc., and you should pause and think by yourself
how you would proceed from this point, before turning the page
and reading the rest.

\endsectionblah

\endchapterblah
