\def\chesspiece#1{\textsymfig{#1}}

\problem.
No xadrez, as regras mandam que as posições iniciais
do jogo são as seguintes:
\newgame
$$
\showboard
$$
Um professor de xadrez pediu para seu aluno
na segunda aula de xadrez
colocar as peças brancas nas suas posições iniciais.
O aluno---ele não é o melhor---lembra apenas
as posições de todos os peões brancos (são todos na linha 2).

Sem querendo assumir sua ignorância,
o aluno colocou todas as outras peças na
primeira linha chutando.
\endgraf
\item{(1)} Qual a probabilidade que ele acertou?
\item{(2)} Se o aluno lembrar mais uma regra:
        ``a rainha branca (\chesspiece{q})
        começa num quadradinho branco'',
qual seria a probabilidade dele acertar?

\solution
(1)
O caso que ele acertou é apenas $1$;
então precisamos contar de quantas maneiras podemos colocar o resto das peças
na primeira linha:
$$
\underbrace{\comb 8 2}_{\dsize\text{\chesspiece{r}\chesspiece{r}}}
\underbrace{\comb 6 2}_{\dsize\text{\chesspiece{n}\chesspiece{n}}}
\underbrace{\comb 4 2}_{\dsize\text{\chesspiece{b}\chesspiece{b}}}
\underbrace{\comb 2 1}_{\dsize\text{\chesspiece{q}}}
\underbrace{\comb 1 1}_{\dsize\text{\chesspiece{k}}}
=
\frac
    {8!\stimes  \cancel{6!}\stimes  \cancel{4!}\stimes  \cancel{2!}}
    {\cancel{6!}\stimes 2!\stimes \cancel{4!}\stimes \cancel{2!}\stimes 2!\stimes 2!}
=
\frac
    {8!}
    {2!\stimes 2!\stimes 2!}
= 7!.
$$
A probabilidade que ele acertou então é $\dfrac 1 {7!}$.
\endgraf
(2)
$
2
\ntimes
\dfrac
    1
    {7!}
    =
    \dfrac
    1
    {2520}
$.
\endproblem

