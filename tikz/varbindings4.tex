\tikzset{
    every node/.style={inner sep=0pt,outer sep=0pt}
}

\node[color=red] (q2) at (0,0)              {$\procrustext{para todo\ }$};
\node[color=red, right=0pt of q2]  (q2v)    {$\hphantom n$};
\node[color=red, right=0pt of q2v] (q2c)    {$\procrustext{,\ }$};
\node[color=red, right=0pt of q2c] (e1)     {$\procrustext{se\ }$};
\node[color=red, right=0pt of e1]  (v1)     {$\hphantom n$};
\node[color=red, right=0pt of v1]  (e2)     {$\procrustext{${}\geq{}$}$};
\node[color=red, right=0pt of e2]  (v2)     {$\procrustes N$};
\node[color=red, right=0pt of v2]  (e3)     {$\procrustext{\ então\ }$};
\node[           right=0pt of e3]  (q3)     {$\procrustext{existe\ }$};
\node[           right=0pt of q3]  (q3v)    {$\hphantom d$};
\node[           right=0pt of q3v] (q3c)    {$\procrustext{\ tal que\ }$};
\node[           right=0pt of q3c] (v4)     {$\hphantom n$};
\node[           right=0pt of v4]  (e4)     {$\procrustes {}-{}$};
\node[           right=0pt of e4]  (v5)     {$\hphantom d$};
\node[           right=0pt of v5]  (e5)     {$\procrustext{\ e\ }$};
\node[           right=0pt of e5]  (v6)     {$\hphantom n$};
\node[           right=0pt of v6]  (e6)     {$\procrustes {}+{}$};
\node[           right=0pt of e6]  (v7)     {$\hphantom d$};
\node[           right=0pt of v7]  (e7)     {$\procrustext{\ são primos}$};

\draw            (q3v.north) edge[out=-80,in=-100,Circle-Circle]  (v5.north);
\draw            (q3v.north) edge[out=-90,in=-90,-Circle]         (v7.north);
\draw[color=red] (q2v.north) edge[out=-70,in=-110,Circle-Circle]  (v1.north);
\draw[color=red] (q2v.north) edge[out=-70,in=-110,-Circle]        (v4.north);
\draw[color=red] (q2v.north) edge[out=-70,in=-110,-Circle]        (v6.north);

