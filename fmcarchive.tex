%%{{{ [vim] 
% vim:foldmarker=%{{{,%}}}
% vim:foldmethod=marker
% vim:foldcolumn=7
%%}}}

%% fmcarchive.tex
%% author: Thanos Tsouanas <thanos@tsouanas.org>
%% Copyright (c) 2022--2023 Thanos Tsouanas
%% All rights reserved.

%%{{{ chapter: Mathematical_logic 
\chapter Lógica matemática.
%%%{{{ meta 
\label Mathematical_logic
%%%}}}

%%{{{ A_propositional_language 
\section Uma linguagem proposicional.
%%%{{{ meta 
\label A_propositional_language
%%%}}}

%%{{{ atomic_vs_composite_prop 
\note Atômicas \vs compostas.
%%%{{{ meta 
\label atomic_vs_composite_prop
%%%}}}

Queremos desenvolver uma linguagem para escrever proposições do tipo que
encontramos em matemática.
A idéia é que as proposições ou são \dterm{atómicas}
(diretamente afirmando algo) ou \dterm{compostas} por outras
sub-proposições que conectamos com os \dterm{conectivos lógicos}:

%%}}}

%%{{{ logical_connectives_prop 
\note Conectivos lógicos.
%%%{{{ meta 
\label logical_connectives_prop
%%%}}}

Considere os seguintes
$$
\xalignat2
\text{conjuncção}:    && (A \land B)       &\qqtext{significa} \textwq{$A$ e $B$} \\
\text{disjuncção}:    && (A \lor B)        &\qqtext{significa} \textwq{$A$ ou $B$} \\
\text{implicação}:    && (A \limplies B)   &\qqtext{significa} \textwq{se $A$ então $B$} \\
\text{equivalência}:  && (A \liff B)       &\qqtext{significa} \textwq{$A$ se e somente se $B$} \\
\text{negação}:       && \lnot A           &\qqtext{significa} \textwq{não $A$}.
\endxalignat
$$
onde os $A,B$ denotam proposições.
O que cada uma dessas cinco proposições significa mesmo, vamos deixar para depois.\foot
Também deixamos para depois a pergunta ainda mais interessante:
o que significa responder na pergunta \wq{O que significa $(A \lor B)$?}?
\toof
Considero que o leitor já tem uma idéia básica sobre o significado de cada uma
dessas frases, e também considero que provavelmente tem uns mal-entendidos também,
e logo não precisa se preocupar agora com seu significado.
Analizamos cada um deles logo no~\ref[Proofs] (e aprofundamos ainda mais nos
\reftag[Mathematical_logic] e \reftag[Proof_theory]).
Agora quero discutir só a sintaxe: \emph{como definir uma linguagem cujo
objetivo é representar esse tipo de proposições}.

%%}}}

%%{{{ examples_of_propositions 
\note.
%%%{{{ meta 
%%%}}}

Considere as seguinte proposições:
% TODO: fix reflabs
\elist:
\li: Se $x < y$ e $y < z$ então $z < x$.
\li: $\sqrt 2$ é irracional.
\li: $5$ é primo
\li: $5$ é primo ou par
\li: $5$ é primo ou $8$ é primo
\li: O $5$ é um divisor dos $25$ e $26$.
\li: Se $p$ é primo então: $p$ divide $8$ se e somente se $p$ é par
\li: Eu não sei onde $p$ nasceu.
\endelist

%%}}}

%%{{{ Its syntax 
\note Sua sintaxe.
%%%{{{ meta 
%%%}}}

\TODO Escrever, mencionar $\zolang$ ou retirar a notação usada logo depois.

%%}}}

%%{{{ Givens 
\note Variáveis proposicionais.
%%%{{{ meta 
\indexes
    * metavariável
    * variável!proposicional
    * variável
    ;;
%%%}}}

Considere dado o conjunto dos símbolos
$$
\zolvars = \set{ \lsym p_0, \lsym p_1, \lsym p_2, \dotsc }
$$
chamados \dterm{variáveis proposicionais}.
Não importa quais são esses símbolos; o que ímporta é que temos
uma infinidade deles, e planejamos usá-los
para representar proposições atômicas.
Esses símbolos então são \emph{variáveis da linguagem-objeto}.
E vamos usar as \emph{metavariáveis} $p,q,r,\dots$ para representá-los.
(Lembre da~\ref[Language_vs_metalanguage]).
Observe que metavariáveis diferentes não necessariamente representam
variáveis diferentes, mesmo que muitas vezes isso é implícito pelo contexto.
Por exemplo, querendo traduzir a proposição
$$
\textwq{$5$ é primo ou par}
$$
podemos escolher atribuir os significados
$$
\rightbrace {
\aligned
\lsym p_0 & : \textwq{$5$ é primo} \\
\lsym p_1 & : \textwq{$5$ é par}
\endaligned
}
\qqtext{e logo usar}
\lsym p_0 \lor \lsym p_1
$$
mas também poderiamos ter escolhido
$$
\rightbrace {
\aligned
\lsym p_9 & : \textwq{$5$ é primo} \\
\lsym p_2 & : \textwq{$5$ é par}
\endaligned
}
\qqtext{e logo usar}
\lsym p_9 \lor \lsym p_2
$$
etc.
Mesmo quando as escolhas são poucas (como aqui),
\emph{preferimos usar metavariáveis} escrevendo
$$
\rightbrace {
\aligned
p & : \textwq{$5$ é primo} \\
q & : \textwq{$5$ é par}
\endaligned
}
\qqtext{e logo usar}
p \lor q
$$
pois \emph{não importa qual} dos símbolos escolhemos para a proposição \wq{$5$ é primo}
e qual para a \wq{$5$ é par}, o importante é que temos dois deles e conseguimos formar a expressão que queremos.

%%}}}

\TODO
Começar com duas definições erradas:
uma sem parens outra com conectivos apenas entre atômicas.

%%{{{ blah 
\blah.
%%%{{{ meta 
%%%}}}

Finalmente chegamos na definição correta do que é uma fórmula:

%%}}}

%%{{{ df: propositional_formula 
\definition Fórmula.
%%%{{{ meta 
\label propositional_formula
\defines
    * fórmula!atômica
    * fórmula!proposicional
    ;;
%%%}}}

\tlist:
\li (A): Se $p$ é uma variável proposicional, então $p$ é uma fórmula.
\li (N): Se $F$ é uma fórmula, então $\lnot F$ é uma fórmula.
\li (B): Se $F,G$ são fórmulas, então:
         \tlist: \withtag --
         \li: $(F \limplies G)$ é uma fórmula;
         \li: $(F \lor G)$ é uma fórmula;
         \li: $(F \land G)$ é uma fórmula.
         \endtlist
\endtlist
Nada mais é uma fórmula.
Uma fórmula que consiste em apenas uma variável proposicional é chamada
\dterm{fórmula atômica}.

%%}}}

%%{{{ note: zolang_bnf 
\note.
%%%{{{ meta 
\label zolang_bnf
%%%}}}

Escrevemos
\mathcol
F &\bnfeq A \bnfor \lnot F \bnfor (F \limplies F) \bnfor (F \land F) \bnfor (F \lor F) \\
\intertext {onde $A$ denota as fórmulas atômicas da $\zolang$, ou seja as variáveis proposicionais:}
A &\bnfeq \lsym p_0 \bnfor \lsym p_1 \bnfor \lsym p_2 \bnfor \dotsb
\endmathcol

%%}}}

%%{{{ x: metavar_vs_vats_in_propositional_formula 
\exercise.
%%%{{{ meta 
%%%}}}

No texto da \ref[propositional_formula], identifique todas as metavariáveis
e todas as variáveis (veja~\reftag[Language_vs_metalanguage], \reftag[metavariables]).


%%}}}

%%{{{ Sugar 
\note Açúcar.
%%%{{{ meta 
%%%}}}

Se $F,G$ são fórmulas, usamos as seguintes abreviações:
$$
\align
(F \liff G)     &\sugareq ((F \limplies G) \land (G \limplies F))\\
(F \limplied G) &\sugareq (G \limplies F)
\endalign
$$

%%}}}

%%{{{ x: second_character_of_limplied_abbr 
\exercise.
%%%{{{ meta 
\label second_character_of_limplied_abbr
%%%}}}

Qual é o segundo caráter da fórmula $(A \limplied B)$?

\hint
Lembre-se que aqui $A,B$ são apenas \emph{metavariáveis}
que denotam algumas fórmulas, sobre quais não sabemos nada mais
fora do fato que são fórmulas (bem formadas).

\solution
O primeiro caráter da fórmula $B$.

%%}}}

%%{{{ Parentheses 
\note Parenteses.
%%%{{{ meta 
%%%}}}

Nos permitimos esquecer as parenteses mais externas
duma fórmula.
$$
F \limplies G \abbreq (F \limplies G).
$$

%%}}}

%%{{{ Precedences 
\note Precedências.
%%%{{{ meta 
%%%}}}

Não vamos escolher nenhuma das $\land$, $\lor$ para considerá-la
mais forte, ou seja, \emph{nunca} vamos escrever algo do tipo
$$
F \lor G \land H.
$$
Por outro lado é comum considerar que elas têm precedência contra
a $\limplies$: com as
$$
F \limplies G \land H
\qqtext{e}
F \lor G \limplies H
$$
denotamos as fórmulas
$$
(F \limplies (G \land H))
\qqtext{e}
((F \lor G) \limplies H)
$$
respectivamente.
Mesmo assim, eu vou tentar botar essas parenteses quando considero
que ajudam na leitura da fórmula!

%%}}}

%%{{{ Associativities 
\note Associatividades.
%%%{{{ meta 
\indexes
    * associatividade!sintáctica
    ;;
%%%}}}

Atribuimos às $\land$ e $\lor$ uma associatividade à esquerda,
e a $\limplies$ uma associatividade à direita.
Pela semântica desejada a primeira escolha é arbitraria,
pois a conjuncção e a disjuncção são operações \emph{associativas}.
Mas a implicação não é.  Nossa escolha então importa!

%%}}}

%%{{{ Its semantics 
\note Sua semântica.
%%%{{{ meta 
%%%}}}

Definimos então a \emph{sintaxe} da nossa linguagem.
Mas por enquanto nenhuma das suas expressões válidas
tem significado!
A gente deixou claro o que é uma fórmula, mas não
o que ela \emph{denota}.
Neste capítulo não vamos definir formalmente alguma
\emph{semântica} para nossa linguagem.
Pelo contrário, usando exemplos e umas explicações
\emph{informais}, eu vou dar uma primeira idéia do que
todas essas fórmulas significam.
O importante é conseguir traduzir uma afirmação escrita
em portugues numa fórmula dessa linguagem, pois assim
ficará mais clara a \emph{estrutúra lógica} que tá
escondida atrás das palavras (e suas ambigüidades)
da linguagem natural.
Depois de bastante trabalho vamos voltar para a questão
de \emph{definir formalmente uma semântica}
para essa linguagem.
Essa tarefa vai nos preocupar nos capítulos
\reftag[Mathematical_logic] e \reftag[Proof_theory].
Por enquanto, deixe pra lá!

%%}}}

\endsection
%%}}}

%%{{{ ZOL_limitations 
\section Limitações da linguagem de proposições.
%%%{{{ meta 
\label ZOL_limitations
%%%}}}

%%{{{ related propositions seem unrelated 
\note.
%%%{{{ meta 
%%%}}}

Considere a afirmação
\quote
Se Igor é brasileiro ele toca violão.
\endquote
Qual é a melhor maneira para traduzir essa afirmação na
nossa linguagem de lógica proposicional?
Estamos procurando então uma \emph{fórmula} da $\zolang$
que poderia representar essa afirmação.
Podemos usar a fórmula seguinte:
$$
(P_0 \limplies P_1)
$$
mas precisamos ainda declarar quais são as proposições
denotadas por as variáveis proposicionais que usamos ($P_0$ e $P_1$).
A escolha é óbvia:
$$
\align
P_0 &:\ \text{Igor é brasileiro}\\
P_1 &:\ \text{Igor toca violão}
\endalign
$$
Note aqui que $P_0$ não poderia ser ``ele toca violão'', pois
na afirmação original esse ``ele'' refere ao Igor, então seria
errado mudar isso para o indefinido e ambiguo ``ele''.
Tudo bem até agora, mas como vamos traduzir as frases:
% TODO: fix reflabs
\tlist:
\li (1): Se Thanos é brasileiro ele toca violão.
\li (2): Se Igor é brasileiro ele toca piano.
\li (3): Se Igor é grego ele toca violão.
\endtlist
Note que cada uma dessas proposições afirma algo muito parecido
com a original, e mesmo assim o melhor jeito que temos
para representar cada uma delas, é:
$$
\align
&(P_2 \limplies P_3)\\
&(P_0 \limplies P_4)\\
&(P_5 \limplies P_1)
\endalign
$$
O problema é que perdemos muita informação nessa traduação,
e olhando apenas para essas fórmulas não conseguimos ver
as conexões que alguém vê lendo as afirmações escritas na
lingua natural acima.

%%}}}

%%{{{ zolang_is_way_too_poor_for_some_propositions 
\note.
%%%{{{ meta 
%%%}}}

Até pior, vamos tentar traduzir a afirmação seguinte:
\quote
Todo brasileiro que fala grego toca piano.
\endquote

%%}}}

%%{{{ Q: what is the best way you can translate this to zolang? 
\question.
%%%{{{ meta 
%%%}}}

Qual é a melhor forma que você consegue traduzir essa proposição para a linguagem
$\zolang$?

%%}}}

\spoiler

%%{{{ a_wrong_translation_to_zolang 
\note.
%%%{{{ meta 
%%%}}}

Uma tentativa errada seria pensar numa fórmula como
$$
((B \land G) \limplies P)
$$
onde escolhendo as proposições denotadas por nossos
$B$, $G$, e $P$, acabamos falando algo sem sentido:
$$
\align
B &: \text{todo brasileiro}\\
G &: \text{alguém que fala grego}\\
P &: \text{ele toca piano}
\endalign
$$
ou algo parecido com isso.
Agora pare e resolva o exercício seguinte:

%%}}}

%%{{{ x: what_is_wrong_with_this_translation_to_zolang 
\exercise.
%%%{{{ meta 
%%%}}}

Qual o problema com essa tradução?

%%}}}

%%{{{ we accept the best is the worst 
\note.
%%%{{{ meta 
%%%}}}

Infelizmente temos que aceitar que a melhor tradução que conseguimos
é denotar a afirmação inteira por alguma variável proposicional.
Ou seja, pelos olhos da $\zolang$, essa é uma afirmação atômica.

%%}}}

%%{{{ x: zolang_limitation_examples 
\exercise.
%%%{{{ meta 
\label zolang_limitation_examples
%%%}}}

Para um exemplo ``mais matemático'', considere as proposições:
\elist:
\li: Para todo inteiro $n$, se $n$ é primo e $n>2$, então $n+2$ é primo.
\li: Existem $n > 2$ e inteiros positivos $a,b,c$, tais que $a^n + b^n = c^n$.
\li: Uma linguagem é regular se e somente se ela é reconhecida por um autômato finito.
\li: Um espaço topológico é normal se e somente se quaisquer dois conjuntos disjuntos e fechados podem ser separados por uma funcção contínua.
\li: Para todo $\alpha\neq0$ algébrico, $e^\alpha$ é transcendental.
\endelist
Qual é a melhor tradução de cada uma dessas na linguagem $\zolang$?

%%}}}

\endsection
%%}}}

%%{{{ FOLs 
\section Linguagens de predicados.
%%%{{{ meta 
\label FOLs 
%%%}}}

%%{{{ the alphabet 
\note O alfabeto.
%%%{{{ meta 
\defines
    * FOL
    ;;
%%%}}}

Pelas limitações da linguagem da lógica proposicional que encontramos
já na~\reftag[ZOL_limitations] faz sentido adicionar pelo menos os dois
símbolos que correspondem nos quantificadores ``para todo'' e ``existe'',
além do resto dos símbolos que já temos:
$$
\lnot, \limplies, \land, \lor, \forall, \exists
$$
\eop
O que mais vamos precisar?
Com certeza precisamos uma nova infinidade de símbolos para \emph{variáveis}
$$
\lsym x_0, \lsym x_1, \lsym x_2, \dots
$$
Essas variáveis não tem nada a ver com as variáveis proposicionais,
pois elas não denotam proposições, mas objetos (indivíduos).
\eop
Que mais?
Faria sentido permitir símbolos de \emph{nomes} ou \emph{constantes}
para denotar certos indivíduos.  Por exemplo, estudando números,
$3$ e $\pi$ são constantes, símbolos que assim que determinar uma
linguagem, eles vão sempre denotar o mesmo objeto.
Observe que não faz sentido falar <<existe $3$ tal que\dots>>
nem <<para todo $3$ \dots>>.  Nossa sintaxe deve proibir essas
abominagens.\foot
Mesmo assim, isso já aconteceu na turma do meu amigo Fagner:
corrigindo provas de cálculo ele viu a frase ``$\forall 3 > 0$''.
Um aluno que, enquanto colando na prova, provavelmente pensou
que seu colega tinha errado no ``$\forall \epsilon > 0$'' escrevendo
o $3$ de cabeça pra baixo.
Depois dele, muitos mais alunos acabaram entregando essa prova com
``$\forall 3 > 0$''.  Ai ai\dots
\toof
Adicionamos então símbolos para constantes
$$
c,d,e,\dots
$$
\eop
Que mais?
Bem, se é para conseguir \emph{formular afirmações sobre objetos},
vamos precisar símbolos para \emph{predicados} que recebem argumentos.
E com eles vamos substituir os símbolos de variáveis proposicionais.
Já discutimos (\reftag[ZOL_limitations]) que uma das limitações que
queremos atender com essa nova linguagem é que afirmações bem parecidas
acabam sendo denotadas por fórmulas que não tem nada em comum:
nossa melhor tradução das afirmações \wq{$8$ é múltiplo de $2$}
e \wq{$10$ é múltiplo de $3$} é algo do tipo $\lsym p_0$ para uma e
$\lsym p_1$ para a outra.
Não serve.  Queremos conseguir usar algo do tipo $P(8,2)$ para uma
e $P(10,3)$ para a outra, onde o $P$ \emph{é o mesmo símbolo},
preservando assim a conexão entre as afirmações.
Jogamos fora então todos os símbolos de variáveis proposicionais
para usar símbolos de \emph{predicados}:
$$
P, Q, R, \dots
$$
e bora botar a virgula \symq{$,$} também para separar os argumentos.
\eop
Algo mais?
Sim, precisamos de uma última coisa.
Símbolos para denotar \emph{funcções}.
Eles nos permitem apontar para algum individual dados outros.
Por exemplo \wq{a mãe da Bárbara}; ou $\lsym x_0 + \sin(\pi)$
que seria \wq{a soma do $\lsym x_0$ com o seno do $\pi$}, etc.
Aqui a gente usaria símbolos como $\lsym{mother}$, $+$, $\ntimes$,
$\sin$, $\cos$, nos números.
Vamos adicionar então símbolos de \emph{funcções}
$$
f, g, h, \dots
$$
e pronto.
Então o alfabeto duma \dterm{linguagem da primeira ordem}
ou \dterm{FOL}, parece assim:
$$
\def\symsnames##1##2{\tubrace{\;\vphantom {g_g} ##1\;} {##2}}
\symsnames {\lnot\ \limplies\ \land\ \lor\ \forall\ \exists\;} {lógica}\ 
\symsnames {(\ )\ ,\;} {puntuação}\ 
\symsnames {\lsym x_0\ \lsym x_1\ \lsym x_2\ \dots\;} {variáveis}\ 
\symsnames {c\ d\ e\ \dots\;} {constantes}\ 
\symsnames {f\ g\ h\ \dots\;} {funcções}\ 
\symsnames {P\ Q\ R\ \dots\;} {predicados}.
$$

%%}}}

%%{{{ givens_for_a_FOL 
\note Os dados para uma linguagem de FOL.
%%%{{{ meta 
%%%}}}

Um conjunto infinito de símbolos de variáveis
$$
\folvars  = \set{\lsym x_0, \lsym x_1, \lsym x_2, \lsym x_3, \dotsc}.
$$
Conjuntos de símbolos de constantes, de funcções, e de predicados:
$$
\folcons,\quad
\folfuns,\quad
\folpreds.
$$
Vamos separar os símbolos de funcções por \dterm{aridade},
ou seja, pela quantidade de argumentos que ele necessita ``consumir''.
Similarmente para os símbolos de predicados, escrevendo assim
$\folfuns_n$ e $\folpreds_n$, para o conjunto de símbolos de funcções
e de predicados de aridade $n$ respectivamente.

%%}}}

%%{{{ remark: nullary functions as constants 
\remark.
%%%{{{ meta 
%%%}}}

Podemos usar funcções de aridade $0$ como constantes,
e assim nem precisamos um conjunto especial para esses símbolos.
Mesmo assim, não vamos fazer isso aqui.
Questão de gosto.\foot
Mentira.
Tem uns detalhes inessenciais neste momento sobre essa escolha.
Vamos discutir isso no~\ref[Mathematical_logic].
\toof

%%}}}

%%{{{ remark: alternative approach with arity metafunction 
\remark.
%%%{{{ meta 
%%%}}}

Alternativamente, podemos deixar os símbolos de aridades
diferentes juntos e exigir uma funcção $\folarity$ que
atribua a cada deles sua aridade.
Por exemplo, numa FOL que usariamos para cálculo, faria
sentido ter símbolos $\lsym +,\lsym{cos},\lsym{<}$ entre outros, com aridades
$\folarity(\lsym{+}) = 2$, $\folarity(\lsym{cos}) = 1$,
$\folarity(\lsym{<}) = 2$, etc.
As duas abordagens são equivalentes:
$$
\align
f \in \folfuns_n  &\iff f \in \folfuns  \mland \folarity(f) = n\\
P \in \folpreds_n &\iff P \in \folpreds \mland \folarity(P) = n.
\endalign
$$
Vamos discutir mais sobre isso no~\ref[Mathematical_logic].

%%}}}

%%{{{ eg: FOL_example_number_theory 
\example.
%%%{{{ meta 
\label FOL_example_number_theory
%%%}}}

Para estudar teoria dos números alguém poderia escolher essa linguagem:
$$
\align
\folcons    &= \set{\lsym 0, \lsym 1, \lsym 2, \dots} \\
\folfuns_1  &= \set{{-}} \\
\folfuns_2  &= \set{{+}, {\cdot}, \lsym{gcd}} \\
\folpreds_1 &= \set{\lsym{Prime}, \lsym{Even}, \lsym{Odd}, \lsym{IsZero}} \\
\folpreds_2 &= \set{{=}, {<}, {\leq}, {\divides}}
\endalign
$$
Note que aqui o símbolo \symq{$-$} foi escolhido para ter aridade $1$.
O usuário dessa linguagem vai usá-lo para representar a operação
unária de negação de números, para formas termos como os:
$-1$, $-(1 + x)$, etc.

%%}}}

%%{{{ eg: FOL_example_people 
\example.
%%%{{{ meta 
\label FOL_example_people
%%%}}}

Para estudar pessoas e suas relações uma linguagem poderia ser:
$$
\align
\folcons    &= \set{\lsym{Thanos}, \lsym{Ramile}, \lsym{Eva}, \lsym{Sam}, \lsym{Maroui}, \lsym{Jebinos}} \\
\folfuns_1  &= \set{\lsym{mother}, \lsym{father}} \\
\folpreds_1 &= \set{\lsym{Male}, \lsym{Female}, \lsym{Adult}, \lsym{Mother}, \lsym{Father}, \lsym{Brazilian}, \lsym{Greek}} \\
\folpreds_2 &= \set{=, \lsym{Loves}, \lsym{SiblingOf}, \lsym{MotherOf}, \lsym{FatherOf}}
\endalign
$$
Qual a diferença entre os símbolos $\lsym{mother}$ e $\lsym{Mother}$?
Ambos são símbolos de aridade $1$, mas $\lsym{mother}$ é um símbolo de funcção
e $\lsym{Mother}$ é um símbolo de predicado.
O que isso quis dizer?
Suponha que $p$ é qualquer objeto do meu mundo (vamos pensar em pessoas).
A expressão $\lsym{mother}(p)$ vai acabar denotando um objeto do meu mundo
também.  A intenção do criador dessa linguagem provavelmente foi que
$\lsym{mother}(p)$ denota a mãe de $p$.
Similarmente, $\lsym{mother}(\lsym{mother}(p))$ denota a mãe da mãe
da pessoa $p$.  E por aí vai.
Por outro lado, a expressão $\lsym{Mother}(p)$ não denota pessoa, mas
sim uma afirmação \emph{sobre} a pessoa $p$.  Adivinhando novamente
a intenção do criador, a proposição $\lsym{Mother}(p)$ afirma que
a pessoa $p$ \emph{é uma mãe}.
E o símbolo $\lsym{MotherOf}$?
Como ele tem aridade $2$, ele precisa de $2$ objetos, e assim
que recebé-los ele denota uma afirmação (pois é símbolo de \emph{predicado}).
Aqui a idéia é que $\lsym{MotherOf}(p,q)$ denota a afirmação
<<$p$ é a mãe de $q$>>.

%%}}}

%%{{{ eg: full_FOL_example 
\example.
%%%{{{ meta 
%%%}}}

Tomamos
$$
\xalignat2
\bigparen{\;\folcons =\;}\quad
\folfuns_0   &= \set{\lsym f^0_0, \lsym f^0_1, \lsym f^0_2, \lsym f^0_3, \dotsc} & \folpreds_0  &= \set{\lsym P^0_0, \lsym P^0_1, \lsym P^0_2, \lsym P^0_3, \dotsc}\\
\folfuns_1   &= \set{\lsym f^1_0, \lsym f^1_1, \lsym f^1_2, \lsym f^1_3, \dotsc} & \folpreds_1  &= \set{\lsym P^1_0, \lsym P^1_1, \lsym P^1_2, \lsym P^1_3, \dotsc}\\
\folfuns_2   &= \set{\lsym f^2_0, \lsym f^2_1, \lsym f^2_2, \lsym f^2_3, \dotsc} & \folpreds_2  &= \set{\lsym P^2_0, \lsym P^2_1, \lsym P^2_2, \lsym P^2_3, \dotsc}\\
             &\eqvdots                                                                   &              &\eqvdots                                                                  
\endxalignat
$$
onde temos denotado a aridade de cada símbolo como um ``exponente''.

%%}}}

%%{{{ Equality 
\remark Igualdade.
%%%{{{ meta 
%%%}}}

Note que podemos considerar o símbolo \symq{$=$} de
igualdade como símbolo da lógica, botando junto com os
$$
\lnot, \limplies, \land, \lor, \forall, \exists, =
$$
e nesse caso falamos que temos uma FOL \emph{com igualdade}.
Ou podemos considerar \symq{$=$} como mais um predicado,
cuja existência ou não na linguagem depende do usuário,
e---ainda mais importante---cuja \emph{interpretação}
também depende do usuário.
Nesse primeiro encontro com lógica e fórmulas, vamos considerar
que $=$ é um símbolo de predicado de aridade $2$, cujo significado
é sempre afirmar que seus $2$ argumentos \emph{denotam o mesmo objeto},
e que sempre faz parte do $\folpreds_2$.
Quando a gente voltar para estudar lógica matemática no
\ref[Mathematical_logic], vamos discutir mais sobre isso.
Por enquanto não importa!

%%}}}

%%{{{ Terms vs. formulae 
\note Terms \vs fórmulas.
%%%{{{ meta 
%%%}}}

Não vamos dar diretamente uma definição de quando uma expressão é uma fórmula.
Em vez disso, vamos primeiro definir o que são os \emph{termos} duma FOL.
A idéia é que o termos representam os objetos (indivíduos) do nosso universo.
Se estamos estudando pessoas, cada termo vai acabar apontando (denotando)
uma pessoa.  Estudando cálculo, os termos denotariam números reais, etc.
Um termo então é um caso especifico de fórmula?
Não!  Temos um ``type error'' se pensar assim!
Os termos denotam objetos.
As fórmulas denotam proposições (afirmações)
(possivelmente \emph{sobre} objetos).
Talvez uns exemplos ajudem.

%%}}}

%%{{{ eg: terms_vs_formulas_example 
\example.
%%%{{{ meta 
%%%}}}

Estudando cálculo, as expressões seguintes são termos:
$$
\xalignat7
 &0
&&\pi
&&x
&&-1
&&1/2
&&\sin(\pi/x)
&&x\sin(\pi/3) + y\cos(\pi^2)
\endxalignat
$$
Por outro lado, as expressões seguintes são fórmulas:
$$
\xalignat4
&
\gathered
{0 < 1} \land {x < 0} \\
{x^2 = 0} \limplies {x = 0}
\endgathered
&&
\gathered
\lnot\exists y (y^2 > 4) \\
x^2 + y^2 = 1
\endgathered
&&
\gathered
\forall x (0 \leq x^2 0) \\
\exists x \forall y (x + y = 0)
\endgathered
&&
\gathered
\forall y \exists y (x + y = 0) \\
\forall z \paren{{z = 0} \lor \forall w \lnot (w + z = w)}
\endgathered
\endxalignat
$$
Uns dos termos chamamos de atômicos e similarmente umas das fórmulas
chamamos de atômicas.  Tente agora (antes de ver a definição) adivinhar
quais desses termos são atômicos e quais dessas fórmulas são atômicas.

%%}}}

%%{{{ df: FOL_term 
\definition Termo.
%%%{{{ meta 
\label FOL_term
\defines
    * FOL!termo
    ;;
%%%}}}

\tlist:
\li: Se $c \in \folcons$ então $c$ é um termo.
\li: Se $f \in \folfuns_n$ e $t_1,\dotsc,t_n$ são termos, então $f(t_1,\dotsc,t_n)$ é um termo.
\li: Se $x \in \folvars$ então $x$ é um termo.
\endtlist
Nada mais é um termo.
Denotamos o conjunto de termos com $\folterms$.
Resumindo esquematicamente a definição, temos:
$$
\align
x \in \folvars &\implies x \in \folterms\\
c \in \folcons &\implies c \in \folterms\\
\rightbrace {
\aligned
f &\in \folfuns_n\\
t_1,\dotsc,t_n &\in \folterms
\endaligned
}
&\implies f(t_1,\dotsc,t_n) \in \folterms\\
\endalign
$$

%%}}}

%%{{{ df: FOL_atomic_formula 
\definition Fórmula atômica.
%%%{{{ meta 
\label FOL_atomic_formula
\defines
    * FOL!fórmula atômica
    ;;
%%%}}}

Se $P \in \folpreds_n$ e $t_1,\dotsc,t_n$ são termos,
então $P(t_1,\dotsc,t_n)$ é uma \dterm{fórmula atômica}.

%%}}}

%%{{{ df: FOL_formula 
\definition Fórmula.
%%%{{{ meta 
\label FOL_formula
\defines
    * FOL!fôrmula
    ;;
%%%}}}

\tlist:
\li (A): Se $A$ é uma fórmula atômica, então $A$ é uma fórmula.
\li (N): Se $F$ é uma fórmula, então $\lnot F$ é uma fórmula.
\li (B): Se $F,G$ são fórmulas, então:
         \tlist: \withtag --
         \li: $(F \limplies G)$ é uma fórmula;
         \li: $(F \lor G)$ é uma fórmula;
         \li: $(F \land G)$ é uma fórmula.
         \endtlist
\li (Q): Se $F$ é uma fórmula e $x$ é uma variável, então:
         \tlist: \withtag --
         \li: $\forall x F$ é uma fórmula;
         \li: $\exists x F$ é uma fórmula.
         \endtlist
\endtlist

%%}}}

%%{{{ Practical abuse of BNF for FOL 
\grammar FOL.
%%%{{{ meta 
\label FOL_grammar
%%%}}}

Seguindo uma práctica comum, escrevemos apenas
$$
\align
F &\bnfeq
\mathit{A}
\bnfor \lnot F
\bnfor (F \limplies F)
\bnfor (F \land F)
\bnfor (F \lor F)
\bnfor \forall v F
\bnfor \exists v F
\endalign
$$
onde $A$ denota fórmulas atômicas e $v$ denota qualquer variável da nossa FOL.

%%}}}

%%{{{ x: BNF_for_FOL 
\exercise.
%%%{{{ meta 
\label BNF_for_FOL
%%%}}}

Supondo que $\bnf{At}$ pode ser substituito por qualquer fórmula atômica duma FOL (e por nada mais),
defina umas gramáticas em BNF para gerar a linguagem das suas fórmulas bem formadas.

\solution
Aqui uma solução:
$$
\align
\bnf{F}     &\bnfeq \bnf{At} \bnfor \lnot \bnf{F} \bnfor (\bnf{F} \bnf{Bin} \bnf{F}) \bnfor \bnf{Q}\bnf{Var}\bnf{F}\\
\bnf{Bin}   &\bnfeq \lor \bnfor \land \bnfor \limplies\\
\bnf{Q}     &\bnfeq \forall \bnfor \exists\\
\bnf{Var}   &\bnfeq x_0 \bnfor x_1 \bnfor x_2 \bnfor \dotsb\\
\intertext{Como já encontramos, um fácil jeito para evitar os ``$\dotsb$'' seria usar:}
\bnf{Var}   &\bnfeq x \bnfor \bnf{Var}'
\endalign
$$

%%}}}

%%{{{ Q: What do we need in order to have a FOL language? 
\question.
%%%{{{ meta 
%%%}}}

O que precisamos especificar para ter uma linguagem FOL?

%%}}}

%%{{{ A: pick symbols and their arities 
\blah Resposta.
%%%{{{ meta 
%%%}}}

Olhando cuidadosamente para a sua sintaxe, é claro que precisamos
deixar claro quais são os símbolos de variáveis, constantes,
funcções, e predicados, que vamos usar, e suas aridades para
os símbolos de funcções e predicados.

%%}}}

%%{{{ Its semantics 
\note Sua semântica.
%%%{{{ meta 
%%%}}}

Tanto como na lógica proposicional, não vamos nos preocupar
neste momento para definir formalmente uma semântica para
as linguagens de lógica da primeira ordem.
Para conseguir isso, precisamos ganhar muita experiência
com conjuntos (\ref[Sets]), funcções (\ref[Functions]),
e relações (\ref[Relations]), e uma certa
\emph{maturidade matemática} que vai chegando durante
nossos estudos.
Finalmente no~\ref[Mathematical_logic] vamos voltar
nessa questão.
Mesmo assim, precisamos dar pelo menos uma primeira idéia
da semântica da FOL, mesmo sem rigidez.

%%}}}

%%{{{ Interpretations 
\note Interpretações.
%%%{{{ meta 
\label Interpretations
\defines
    * interpretação
    ;;
%%%}}}

Dada uma linguagem de FOL, para suas fórmulas ganharem
significado, precisamos deixar claro:
\tlist:
\li: Qual é o \dterm{universo}, ou seja, todos os objetos
onde as variáveis (e constantes) tomam seus valores.
Todos os nossos termos vão acabar denotando objetos
desse universo.
\li: Para cada \emph{símbolo de funcção}, qual é a \emph{funcção mesmo}
que ele representa no universo escolhido.
\li: Para cada \emph{símbolo de predicado}, qual é a \emph{relação mesmo}
que ele representa no universo escolhido.
\endtlist
Assim que der tudo isso, digamos que temos uma \dterm{interpretação}
da linguagem.

%%}}}

%%{{{ Definition of truth 
\note Definição de verdade.
%%%{{{ meta 
%%%}}}

Mas o que significa que uma fórmula é verdadeira?
Tendo uma interpretação (\reftag[Interpretations])%
---ou seja, assim que especificar tudo isso---se numa fórmula
não aparecem variáveis livres, podemos investigar já sua veracidade,
testando a correspondente afirmação na interpretação.
Se aparecem variáveis livres, tendo uma atribuição de objetos
nelas também podemos decidir sua veracidade.
Essa curta explicação informal deve servir por enquanto,
junta com os exemplos abaixo.
No~\ref[Mathematical_logic] vamos ver tudo formalmente e com carinho.
Paciência.

%%}}}

%%{{{ eg: mother_of_all 
\example.
%%%{{{ meta 
\label mother_of_all
%%%}}}

Considere as fórmulas
\elist:
\li: $\forall x (P(x) \limplies Q(x))$;
\li: $\exists x \forall y R(x,y)$.
\li: $\forall y \exists x R(x,y)$.
\endelist
Escolhendo que o universo é feito por todas as pessoas
(tanto vivas quanto mortas) e que $P$ e $Q$ são os predicados de
``ser pai'' e ``ser homem'', a primeira fórmula vira-se verdade:
realmente cada pai é um homem.  Escolhendo interpretar o $R(u,v)$
como ``$u$ é a mãe de $v$'', a segunda frase se-traduza como
``existe pessoa que é mãe de todos'' (que é falso),
mas a terceira diz ``toda pessoa tem mãe'' (que é verdadeiro).

%%}}}

%%{{{ x: interpretting_formulas_exercise
\exercise.
%%%{{{ meta 
\label interpretting_formulas_exercise
%%%}}}

Para cada uma das fórmulas abaixo, quando possível,
ache uma interpretação que a satisfaz e uma que não.
\elist:
\li: $\forall x \forall y \paren{R(x,y) \land \lnot R(y,x)}$;
\li: $\forall x \forall z \exists y \paren{R(x,y) \land R(y,z)}$;
\li: $\exists x \exists y \exists z Q(f(x,y),z)$.
\endelist

%%}}}

%%{{{ Translations to and from FOL 
\note Traduzindo de e para FOL.
%%%{{{ meta 
\label FOL_translations
%%%}}}

\TODO.

%%}}}

%%{{{ Non-limitations 
\note Não-limitações.
%%%{{{ meta 
%%%}}}

Numa primeira olhada a linguagem pode parecer limitada
para expressar umas afirmações que encontramos o tempo
todo em matemática:
$$
\gathered
\text{``existe único $x$ tal que \dots''} \\
\text{``para todo número primo $p$, \dots''} \\
\text{``existe $x\in A$ tal que \dots''} \\
\text{``para todo $x\in A$, \dots''} \\
\text{``existe único $x\in A$ tal que \dots''}
\endgathered
$$
etc.
Podemos expressar essas afirmações numa linguagem de FOL?
Se não, vamos precisar aumentar nossa linguagem
com novos quantificadores, mas felizmente isso
não é necessário.  Podemos realmente escrever
todas essas afirmações e logo basta só descrever
como, e definir mais \emph{acúcar sintáctico}.
Trabalhe nos exercícios seguintes para ver como.

%%}}}

%%{{{ x: forall_x_in_A_sugar 
\exercise.
%%%{{{ meta 
\label forall_x_in_A_sugar
%%%}}}

Ache uma fórmula de linguagem de FOL equivalente à
$\lforall {x \in A} {\phi(x)}$, onde $\phi(x)$ é
uma fórmula que envolve o $x$.  Podes supor que
tua linguagem tem o símbolo \symq{$\in$} nos seus predicados.

\hint
Só para ajudar, suponha que o universo é feito
por todas as pessoas da terra, e que $A$ é o conjunto
de todos os alunos.
E só para ficar mais concreto ainda, suponha
que a fórmula $\phi(x)$ quis dizer que
<<$x$ é uma pessoa legal>>.
O desafio é conseguir escrever a afirmação
$$
\text{<<todo aluno é legal>>}
$$
O problema é que minha linguagem me permite dizer
$$
\forall x {\lthole}
$$
ou seja
$$
\text{<<toda pessoa, \dots>>},
\qquad
\text{<<para toda pessoa $x$, $\psi(x)$>>},
$$
etc., mas não <<para todo aluno $x$, $\psi(x)$>>.
O que você precisa afirmar \emph{para toda pessoa $x$} mesmo,
tal que a afirmação inteira vai acabar sendo equivalente a
$$
\text{<<todo aluno é legal>>}?
$$

\hint
Tu tens que afirmar algo para toda pessoa $x$.
Em vez de afirmar que $x$ é legal, afirme uma implicação:
afirme que
$$
\text{se $x$ {\lthole}, então {\lthole}}.
$$

\solution
$$
\lforall {x \in A} {\phi(x)}
\iff
\forall x \paren{ x\in A \limplies \phi(x) }
$$

%%}}}

%%{{{ x: exists_x_in_A_sugar 
\exercise.
%%%{{{ meta 
\label exists_x_in_A_sugar
%%%}}}

Similarmente para a $\lexists {x \in A} {\phi(x)}$.

\hint
A idéia é parecida com o~\ref[forall_x_in_A_sugar],
mas cuidado, pois é fácil errar!

\hint
Se eu quiser afirmar que existe \emph{um aluno} $x$ tal que $\phi(x)$,
mas eu posso apenas afirmar que existe \emph{uma pessoa} $x$ tal que $\psi(x)$, como eu posso escolher esse $\psi(x)$ para conseguir dizer o que eu quero?
Observe que afirmando que existe aluno $x$ tal que $\phi(x)$,
é a mesma coisa de afirmar que existe pessoa $x$ tal que $\phi(x)$ e\dots
mais uma coisa!
Qual?

\solution
$$
\lexists {x \in A} {\phi(x)}
\iff
\lexists x {x\in A \land \phi(x)}
$$

%%}}}

%%{{{ x: unique_x_sugar 
\exercise.
%%%{{{ meta 
\label unique_x_sugar
%%%}}}

Similarmente para
$\lunique x {\phi(x)}$ e uma da $\lunique {x \in A} {\phi(x)}$.
Lemos o ``$\unique{}$'' como ``existe único''.

%%}}}

%%{{{ x: is_this_sugar_good_for_unique_1 
\exercise.
%%%{{{ meta 
\label is_this_sugar_good_for_unique_1
%%%}}}

$
\lunique x {\phi(x)}
\askiff
\pexists u \lforall y { \phi(y) \liff y = u }
$

%%}}}

%%{{{ x: is_this_sugar_good_for_unique_2 
\exercise.
%%%{{{ meta 
\label is_this_sugar_good_for_unique_2
%%%}}}

$
\lunique x {\phi(x)}
\askiff
\lexists x { \phi(x) \mland \lforall y { \phi(y) \limplies \phi(x) } }
$

%%}}}

%%{{{ x: square_of_odd_is_odd 
\exercise.
%%%{{{ meta 
\label square_of_odd_is_odd
%%%}}}

Considere a afirmação:
$$
\textwq{Todos os quadrados de ímpares são ímpares.}
$$
Usando a notação encontrada neste capítulo, escreva a afirmação em tal forma
que sua estrutura lógica é claramente exposta.
Considere o \wq{{\thole} é ímpar} como um predicado primitivo.

\hint
$\lforall {n \in \ints} {\dots?\dots}$.

\hint
$\lforall {n \in \ints} {\text{$n$ ímpar} \askiff \text{$n^2$ ímpar}}$.

\solution
$\lforall {n \in \ints} {\text{$n$ ímpar} \implies \text{$n^2$ ímpar}}$.

%%}}}

%%{{{ Limitations 
\note Limitações.
%%%{{{ meta 
%%%}}}

Para a maioria das afirmações que encontramos em matemática normalmente
a FOL é suficiente.  Mesmo assim, ela tem suas limitações tanto
para expressar certas proposições matemáticas, quanto para certas afirmações
que encontramos com freqüência na nossa vida.

%%}}}

%%{{{ x: limitation_of_fol_in_math 
\exercise.
%%%{{{ meta 
\label limitation_of_fol_in_math
%%%}}}

Ache uma afirmação matemática que FOL não é capaz para traduzir bem.

%%}}}

%%{{{ x: modal_temporal_logics 
\exercise.
%%%{{{ meta 
\label modal_temporal_logics
%%%}}}

Pense em outras afirmações que aparecem freqüentemente na nossa fala
que FOL seria pobre para traduzir bem.

%%}}}

\endsection
%%}}}

%%{{{ PROBLEMS 
\problems.
%%%{{{ meta 
%%%}}}

\endproblems
%%}}}

%%{{{ Further reading 
\further.

Sobre lógica matemática:
\cite[kleeneIM],
\cite[curryfoundations];
\cite[kleenelogic],
\cite[smullyanfol];
\cite[corilascar1]~\&~\cite[corilascar2],
\cite[shoenfieldlogic],
\cite[bellmachover].

Sobre intuicionismo:
\cite[heytingintuitionism].
\cite[dummettintuitionism].
\cite[proofsandtypes].
\cite[lecturesch].
\cite[bishopfca]~\&~\cite[bishopbridgesconstructive].
\cite[girardblindspot].

\endfurther
%%}}}

\endchapter
%%}}}

